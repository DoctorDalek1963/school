\documentclass[a4paper, 12pt]{article}

\usepackage[UKenglish]{babel} % UK writing style
\usepackage[a4paper, top=15mm, bottom=25mm, left=30mm, right=30mm]{geometry}

\usepackage[hidelinks]{hyperref}
\hypersetup{
	pdftitle = {Dyson - draft \#NUM},
	pdfauthor = {Dyson}
}

\usepackage{csquotes}
\usepackage{setspace}

\usepackage{xcolor}
\newcommand{\todo}[1]{\textsl{\textcolor{gray}{(TODO: #1)}}}

\usepackage{luacode}
\begin{luacode*}
do
--[[
	These two functions are used to get the contents of the personalstatement
	environment in the whole_buf string, and then get the number of words and
	chars from that so we can add that information at the end of the environment.

	This actually counts the raw characters of the source code, and a word is
	defined as any substring of non-whitespace characters.
--]]
	local whole_buf = ""

	function readbuf(buf)
		whole_buf = whole_buf .. buf .. "\n"
	end

	function startPersonalStatement()
		whole_buf = ""
		luatexbase.add_to_callback("process_input_buffer", readbuf, "readbuf")
	end

	function stopPersonalStatement()
		luatexbase.remove_from_callback("process_input_buffer", "readbuf")

		local buf = whole_buf:gsub("\n?\\end{personalstatement}\n$", "")

		-- Don't count \todo commands, or extraneous whitespace
		buf = buf:gsub(" *\\todo{[^{}]+}[. ]*", "")

		-- This newlines thing is a hack to remove single newlines but keep double newlines
		buf = buf:gsub("\n", " ")
		buf = buf:gsub("  ", "\n\n")

		print("BUFFER_START")
		print(buf)
		print("BUFFER_END")

		-- string.gsub returns the number of substitutions as
		-- its second return value, so we only catch that one
		local _, chars = buf:gsub(".", "")
		local _, words = buf:gsub("%S+", "")

		local info_string = string.format("\\par\\hfill %s words, %s characters", words, chars)
		tex.sprint(info_string)
	end
end
\end{luacode*}

\newenvironment{personalstatement}{\directlua{startPersonalStatement()}}{\directlua{stopPersonalStatement()}}

\begin{document}

\begin{center}
	\vspace*{3mm}
	\huge{\textbf{Dyson - draft \#NUM}}
\end{center}

\setlength{\parskip}{7.5ex}
\setlength{\parindent}{0em}
\setstretch{2}

\vspace*{-12ex} % Reduce ugly space between title and text
\section*{Main Personal Statement}
\vspace*{-6ex}

\begin{personalstatement}
I am particularly interested in the abstract parts of maths that aren't covered in the A Level curriculum. Things like group theory, category theory, type theory, topology, linear algebra, etc. I especially love visualising these abstract concepts where applicable.

For example, my Computer Science coursework was focused on visualising 2D linear transformations. When I first learned them in school, I struggled to visualise them, so I wanted a tool that would show a given transformation and allow the user to interact with it. An interactive version of the 3blue1brown linear algebra series would be perfect, but I couldn't find anything suitable. I mainly wanted to see matrix multiplication represented as transformation composition. Since I couldn't find a suitable app, I made one myself. During development, I faced many mathematical challenges such as converting between coordinate systems and drawing lines in the correct places on the canvas. It involved solving many interesting geometric problems, and researching various computational methods. The project proved useful for explaining matrices to my peers, and teachers have used it to teach the topic in lessons.

Certain questions in textbooks have prompted me to write my own LaTeX papers to explore the ideas they present. For example, a question which asked for polynomials which give pure powers of n when summed from 1 to n. While exploring the generalisation, I found an interesting pattern in the coefficients that looked like those of binomial expansions. Exploring this pattern, I found a formula. My proof was originally several pages long but then I asked a forum about my discovery and learned that it was simply a finite telescoping series. This made my proof far shorter, and I learned a valuable lesson about stepping back to see simpler solutions, and collaborating with other people.

When I listened to A Brief History of Mathematics on BBC Sounds, my favourite episode was about Galois. It detailed his life and inspired me to look further into Galois theory, although the prerequisite knowledge is quite advanced, so I can only just get the gist. It's incredibly interesting and I definitely want to learn more about it. Abstract algebras like Galois theory, group theory, and quaternions are the topics that interest me the most, since they don't have much obvious practical application, and I want to dive deep to understand them in more detail.

In Year 12, I joined an online course with MIT about using matrices to solve linear differential equations. I didn't know what differential equations were at the time, and I barely understood matrices, but I wanted to extend my knowledge. It started with linear transformations and matrices, and then introduced eigenvectors and eigenvalues. My main takeaways were row echelon form and Gauss-Jordan elimination, which allowed me to easily solve linear simultaneous equations by hand. I supplemented it with the 3blue1brown linear algebra series, but when the course started to use these techniques to solve differential equations, I had to stop because I knew nothing about the subject. I tried to learn about differential equations with another MIT course, but it wasn't very helpful for me, so I had to step away and come back later, although it did teach me valuable lessons about mathematical modelling. I also learned that having a good foundation is vitally important when learning new topics.

I completed the UKMT Senior Challenge and got Gold and Best in School. I took TMUA and the MAT in 2022, and STEP in 2023, although I didn't get the STEP grades that I wanted. I was practising STEP questions a lot, but I often gave up too early and didn't mark those questions, so I didn't know my weaknesses. I did a lot of practice, but it wasn't properly targeted, so I didn't improve quick enough. I have since learned my lesson and I now have a system in place to make sure my STEP practice is actually effective.
\end{personalstatement}

%\vspace*{-6ex}
%\section*{SAQ Personal Statement}
%\vspace*{-6ex}

%\begin{personalstatement}
%I want to be at Cambridge because I know that I would benefit significantly from the environment and teaching style. Being surrounded by professional academic and fellow high-achieving students who love maths just as much as I do will fuel my passion for the subject and keep me working hard. I also know that I learn better in small groups when I can chat comfortably with someone who knows what they're talking about. The supervisions at Cambridge will provide that environment where I feel comfortable asking questions to an academic, and I can better expand my knowledge and learning.
%\end{personalstatement}

\end{document}
