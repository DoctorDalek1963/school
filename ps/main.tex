\documentclass[a4paper, 12pt]{article}

\usepackage[UKenglish]{babel} % UK writing style
\usepackage[a4paper, top=15mm, bottom=25mm, left=50mm, right=20mm]{geometry} % Good margins

\usepackage[hidelinks]{hyperref}
\hypersetup{
	pdftitle = {Dyson - draft \#NUM},
	pdfauthor = {Dyson}
}

\usepackage{csquotes}
\usepackage{setspace}

\usepackage{xcolor}
\newcommand{\todo}[1]{\textsl{\textcolor{gray}{(TODO: #1)}}}

\usepackage{luacode}
\begin{luacode*}
do
--[[
	These two functions are used to get the contents of the personalstatement
	environment in the whole_buf string, and then get the number of words and
	chars from that so we can add that information at the end of the environment.

	This actually counts the raw characters of the source code, and a word is
	defined as any substring of non-whitespace characters.
--]]
	local whole_buf = ""

	function readbuf(buf)
		whole_buf = whole_buf .. buf .. "\n"
	end

	function startPersonalStatement()
		whole_buf = ""
		luatexbase.add_to_callback("process_input_buffer", readbuf, "readbuf")
	end

	function stopPersonalStatement()
		luatexbase.remove_from_callback("process_input_buffer", "readbuf")

		local buf = whole_buf:gsub("\n?\\end{personalstatement}\n$", "")

		-- Don't count \todo commands, or extraneous whitespace
		buf = buf:gsub(" *\\todo{[^{}]+}[. ]*", "")

		-- This newlines thing is a hack to remove single newlines but keep double newlines
		buf = buf:gsub("\n", " ")
		buf = buf:gsub("  ", "\n\n")

		print("BUFFER_START")
		print(buf)
		print("BUFFER_END")

		-- string.gsub returns the number of substitutions as
		-- its second return value, so we only catch that one
		local _, chars = buf:gsub(".", "")
		local _, words = buf:gsub("%S+", "")

		local info_string = string.format("\\par\\hfill %s words, %s characters", words, chars)
		tex.sprint(info_string)
	end
end
\end{luacode*}

\newenvironment{personalstatement}{\directlua{startPersonalStatement()}}{\directlua{stopPersonalStatement()}}

\begin{document}

\begin{center}
	\vspace*{3mm}
	\huge{\textbf{Dyson - draft \#NUM}}
\end{center}

\setlength{\parskip}{7.5ex}
\setlength{\parindent}{0em}
\setstretch{2.5} % A bit more than double spacing

\vspace*{-12ex} % Reduce ugly space between title and text
\section*{Main Personal Statement}
\vspace*{-6ex}

\begin{personalstatement}
Maths has interested me since GCSE and my passion has only grown stronger with time. I love all of A Level maths, but have a particular passion for the more abstract stuff not covered in the normal curriculum. I also really enjoy visualising maths for myself and others.

My Computer Science coursework is focused on visualising mathematics to help teach it. When I was learning linear transformations in school, I struggled to visualise them and I know a lot of my peers also struggled, so I wanted a tool that would show a given transformation and allow you to interact with it. I watched the 3blue1brown linear algebra series and wanted an app inspired by it, but I could not find any that did everything I wanted, which was mainly to see matrix multiplication as transformation composition. Since I could not find a good app, I made one myself. In the course of this development, I faced many mathematical challenges around converting between coordinate systems and drawing lines on the canvas. It has involved solving many small but interesting geometric problems, and doing a lot of research into methods and frameworks. The project has proved useful for explaining matrices to my peers, and teachers have said they will use it to teach the topic next year.

Certain questions in textbooks have prompted me to write my own LaTeX papers to explore the ideas these questions present. For example, there was a question that asked for polynomials which give pure powers of n when summed from 1 to n. While exploring the generalisation, I found an interesting pattern in the coefficients that looked like those of binomial expansions. I explored this pattern and found a formula. My proof was originally several pages long but I then asked StackExchange about my discovery and learned that it was a finite telescoping series. This made the proof far shorter, and I learned a valuable lesson about stepping back to see simpler solutions, and collaborating with other people. I was also doing a lot of research into solutions to similar problems, stumbled on the Bernoulli numbers, and ended up with a whole bibliography for the paper.

I listened to A Brief History of Mathematics on BBC Sounds. My favourite episode was about Galois. It detailed his life and inspired me to look further into Galois Theory. I found two YouTube videos on the topic. It is obviously beyond me right now, but I definitely want to learn more. Abstract algebras like Galois Theory, Group Theory, and the dual numbers are very interesting to me. It is these kind of abstract (and often fundamental) topics that interest me most.

Early in Year 12, I joined an MIT MOOC about using matrices to solve linear differential equations. I did not know what differential equations were, and I barely understood matrices, but I wanted to do the course. It started slow with linear transformations and matrices, and then introduced eigenvectors and eigenvalues. It was a great introduction and taught me a lot about the subject. My main takeaways were row echelon form and Gauss-Jordan elimination, which allowed me to easily solve linear simultaneous equations by hand. I supplemented it with the 3blue1brown linear algebra series. But when the course started to use these techniques to solve differential equations, I had to stop because I knew nothing about the subject. I tried to learn the topic with another MIT course, but this was way over my head. I did not have the knowledge for it at the time, so I had to step away and come back later. I learned that having a good foundation is very important when learning new topics.

I did the UKMT Senior Challenge and got Gold and Best in School. I am doing the MAT and TMUA at the end of 2022, and STEP in mid-2023.
I also attended the Imperial College Further mA*ths Year 12 MOOC, which gave me a better understanding of certain topics and stoked my passion. I will be doing their Year 13 course. I am also part of the Problem Solving Matters course to prepare me for admissions tests.
\end{personalstatement}

\vspace*{-6ex}
\section*{SAQ Personal Statement}
\vspace*{-6ex}

\begin{personalstatement}
I watched Welch Labs' series on complex numbers, which has a great visualisation of multifunctions and 4D Riemann Surfaces. I first watched this series before I had learned about complex numbers in school and only understood the first half of it. I have since rewatched the series and I still do not fully understand all of it, but I understand most of it, and the more advanced parts are very interesting to me. The series is fantastic at communicating complex ideas in a visual way, and has greatly increased my understanding of complex numbers, along with other online learning resources.

Episode 2 of The Numberphile Podcast is an interview with Ken Ribet about Fermat's Last Theorem. This episode is very interesting and gives a good overview of the story and touches on Wiles' proof. However, I wanted to investigate further. I watched some YouTube videos about elliptic curves and modular forms and got a very surface-level overview of the topic. Upon further research, I continued to learn more about it. I still do not understand it in much detail, but I get the gist and I love the connections between number theory and complex analysis. Connecting seemingly disconnected areas of mathematics like this is something I love, just like how I connected polynomials that sum to pure powers with coefficients of binomial expansions.

I want to be at Cambridge because I know I would benefit significantly from the environment and teaching style. Being surrounded by other nerds who love maths just as much as I do will fuel my passion for the subject and keep me working hard. I also know that I learn better in small groups when I can chat almost casually with someone who knows what they're talking about. The supervisions at Cambridge will provide that environment where I feel comfortable asking questions to an academic, and I can better expand my knowledge and learning.
\end{personalstatement}

\end{document}
