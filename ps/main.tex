\documentclass[a4paper, 12pt]{article}

\usepackage[UKenglish]{babel} % UK writing style
\usepackage[a4paper, top=15mm, bottom=25mm, left=50mm, right=20mm]{geometry} % Good margins

\usepackage[hidelinks]{hyperref}
\hypersetup{
	pdftitle = {Dyson - draft \#NUM},
	pdfauthor = {Dyson}
}

\usepackage{csquotes}
\usepackage{setspace}

\usepackage{xcolor}
\newcommand{\todo}[1]{\textsl{\textcolor{gray}{(TODO: #1)}}}

\usepackage{luacode}
\begin{luacode*}
do
--[[
	These two functions are used to get the contents of the personalstatement
	environent in the whole_buf string, and then get the number of words and
	chars from that so we can add that information at the end of the environment.

	This actually counts the raw characters of the source code, and a word is
	defined as any substring of non-whitespace characters.
--]]
	local whole_buf = ""

	function readbuf(buf)
		whole_buf = whole_buf .. buf .. "\n"
	end

	function startPersonalStatement()
		whole_buf = ""
		luatexbase.add_to_callback("process_input_buffer", readbuf, "readbuf")
	end

	function stopPersonalStatement()
		luatexbase.remove_from_callback("process_input_buffer", "readbuf")

		local buf = whole_buf:gsub("\n?\\end{personalstatement}\n$", "")

		-- Don't count \todo commands, or extraneous whitespace
		buf = buf:gsub(" *\\todo{[^{}]+}[. ]*", "")
		buf = buf:gsub("\n", " ")
		buf = buf:gsub("%s{2,}", "")

		-- string.gsub returns the number of substitutions as
		-- its second return value, so we only catch that one
		local _, chars = buf:gsub(".", "")
		local _, words = buf:gsub("%S+", "")

		local info_string = string.format("\\par\\hfill %s words, %s characters", words, chars)
		tex.sprint(info_string)
	end
end
\end{luacode*}

\newenvironment{personalstatement}{\directlua{startPersonalStatement()}}{\directlua{stopPersonalStatement()}}

\begin{document}

\begin{center}
	\vspace*{3mm}
	\huge{\textbf{Dyson - draft \#NUM}}
\end{center}

\setlength{\parskip}{7.5ex}
\setstretch{2.5} % A bit more than double spacing

\vspace*{-12ex} % Reduce ugly space between title and text
\section*{Main Personal Statement}
\vspace*{-6ex}

\begin{personalstatement}
I do Maths, Further Maths, and Computer Science A Levels, which were all chosen to further my educational career in mathematics. I originally took Physics but I was disappointed by the lack of maths in the course, and I dropped it to focus more on pure maths. I also did the UKMT Senior Challenge and got Gold and Best in School. I'm also doing the MAT and TMUA at the end of 2022, and STEP in mid-2023.

I did my Computer Science coursework a year early and the project is entirely focussed on visualizing mathematics for the benefit of teaching. When I was learning linear transformations in school, I struggled to visualize the transformations and I know a lot of my peers also struggled with it, so I wanted a tool that would show a given transformation and allow you to interact with it. I watched the 3blue1brown linear algebra series and wanted an interactive app inspired by the series, but I couldn't find any that did everything I wanted. I mainly wanted to see how matrix multiplication works as transformation composition. Since I couldn't find a good app, I decided to make one myself. In the course of this development, I faced many mathematical challenges around converting between coordinate systems and drawing lines on the canvas. It has involved solving many small but interesting geometric problems. The project has also proved very useful for explaining matrices to other people, and for investigating certain questions involving 2D transformations.

I attended the Imperial College London Further mA*ths Year 12 online course, which gave me a better understanding of many core principles of the subject. These topics included things like Maclaurin series, de Moivre's theorem, and hyperbolic functions, all taught to Year 12s. I will also be doing their Year 13 course.

I've found questions in textbooks that have prompted me to write my own papers using LaTeX to explore the ideas these questions present. A simple example would be a question that asked how many regions can be formed by dividing the plane with n lines. It was an interesting question that I enjoyed solving, but I then wanted to produce a nicely formatted paper using TikZ to draw some illustrative graphics, so I did. A more interesting example of writing my own papers is the question that asked for polynomials which give pure powers of n when summed from 1 to n. While exploring this problem, I found an interesting pattern in the coefficients that looked like binomial expansions. I explored this pattern and found a compact formula. My proof of this formula was originally several pages long but I then asked StackExchange if my discovery had a name and learned that it was a finite telescoping series. This made the proof far shorter, and I learned a valuable lesson about stepping back to see simpler solutions, and collaborating with other people.

I listened to A Brief History of Mathematics on BBC Sounds, which has episodes on various people in mathematical history. My favourite episode was the one on Galois. The episode details his life and inspired me to look further into Galois Theory. I found two YouTube videos on a channel called Aleph 0 about the topic, and they were both very interesting. The topic is obviously beyond me right now, but I think I've got a grasp of the very basics, and I definitely want to learn more. Abstract algebras like Galois Theory, Group Theory, and even things like the dual numbers are incredibly interesting to me. It's these kind of abstract (and often fundamental) algebras that interest me the most.

Early in Year 12, I joined a MOOC with MIT about using matrices and their eigenstuffs to solve linear differential equations. I didn't know what differential equations were, and I barely understood matrices, but I wanted to jump into the course. It started slow with linear transformations and matrices and their relationship and then introduced eigenvalues and eigenvectors. It was actually a very nice introduction and taught me a lot about the subject. My main takeaways were row echelon form and Gauss-Jordan elimination, which allowed me to easily solve linear simultaneous equations by hand. I also supplemented it with the 3blue1brown linear algebra series. However, when the course started to talk about using these techniques to solve differential equations, I had to give up because I knew nothing about the subject. I tried to learn the topic with another MIT course, but this course was way over my head. It felt like it was aimed at people who had a much greater knowledge of calculus. After being defeated by this course, I went back to it several months later. \todo{Finish doing this course and write about it}

I watched Welch Labs' series on complex numbers, which has a fantastic visualization of multifunctions and 4D Riemann Surfaces. I first watched this series before I had learned about complex numbers in school and only understood the first half of it. I have since rewatched the series and I still don't understand all of it, but I've watched enough of Cliff Stoll talking about Klein Bottles to understand a fake intersection caused by projecting 4D into 3D. The series is fantastic at communicating complex ideas in a visual way, and has greatly increased my understanding of complex numbers, along with other online learning resources.

Episode 2 of the Numberphile podcast is an interview with Ken Ribet about Fermat's Last Theorem. This episode is very interesting and gives a good overview of the story and touches on Wiles' proof. However, I wanted to investigate further. I watched some YouTube videos from Aleph 0 about elliptic curves and modular forms and got a very surface-level overview of the topic. Upon further research, I continued to learn more about the topic. I still don't understand it in much detail, but I get the gist and I love the connections between number theory and complex analysis. Connecting seemingly disconnected areas of mathematics like this is something I love, just like how I connected polynomials that sum to pure powers with coefficients of binomial expansions.
\end{personalstatement}

\vspace*{-6ex}
\section*{SAQ Personal Statement}
\vspace*{-6ex}

\begin{personalstatement}
SAQ example personal statement
\end{personalstatement}

\end{document}
